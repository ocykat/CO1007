\documentclass[10pt]{article}

\usepackage{pictex, latexsym, graphicx,amsmath,amssymb,amsbsy,amsfonts,amsthm,verbatim}
\usepackage{graphics}
\usepackage{fullpage}
\usepackage{fancyhdr}

\usepackage{algorithm,algorithmic}
\usepackage{multirow}

\setlength{\voffset}{-0.25in}
\setlength{\headsep}{0.25in}
\setlength{\parskip}{1em}
\setlength{\parindent}{0em}

% Counter for the problems.
\newcounter{problem}
\newcommand{\problem}{\textbf{\refstepcounter{problem}Problem \theproblem.} }

% Macros
\def\vu{\mathbf{u}}
\def\vx{\mathbf{x}}
\def\vb{\mathbf{b}}
\def\vv{\mathbf{v}}
\def\vw{\mathbf{w}}

\renewcommand{\implies}{\rightarrow}
\renewcommand{\lor}{\vee}
\renewcommand{\land}{\wedge}
\newcommand{\xor}{\oplus}
\renewcommand{\iff}{\leftrightarrow}
\newcommand{\TRUE}{\mathbf{T}}
\newcommand{\FALSE}{\mathbf{F}}
\newcommand{\universe}{\mathcal{U}}
\newcommand\R{\mathbb{R}}


% Math commands:
% \forall, \exists

\begin{document}
\pagestyle{fancyplain}
%%%%%%%%%%%%%%%%%%%%%%%%%%%%%%%%%%
\chead{DISCRETE STRUCTURE (CO1007) --- SOLUTION to Homework 02 --- PREDICATE LOGIC \& PROOF}
\textit{\textbf{\underline{Instruction:}} Type your answers to the following questions provided by LaTeX and submit a zipped file (included .pdf and .tex files) to E-learning by group (only 4-5 members in each group). Only team leader will submit it. One page per problem. Please use the solution template provided.}

% ********** GROUP INFO **********

\begin{center}
    \begin{tabular}{|c|c|c|c|}
        \hline 
        \multicolumn{4}{|c|}{\textbf{GROUP ... ---- MEMBER LIST}} \\ 
        \hline 
        \textbf{No.} &\qquad\qquad \textbf{Name}\qquad\qquad\qquad & \qquad\textbf{ID}\qquad\qquad & \qquad\textbf{Role}\qquad\qquad \\ 
        \hline 
        1 &  &  &  \\ 
        \hline 
        2 &  &  &  \\ 
        \hline 
        3 &  &  &  \\ 
        \hline 
        4 &  &  &  \\ 
        \hline 
        5 &  &  &  \\ 
        \hline 
    \end{tabular} 
\end{center}



% ********** PROBLEM 01 **********

%---------------------------------------------------------------------------------------------------------------------------------------------------


% ********** PROBLEM 02 **********

%---------------------------------------------------------------------------------------------------------------------------------------------------



% ********** PROBLEM 03 **********

%---------------------------------------------------------------------------------------------------------------------------------------------------



% ********** PROBLEM 04 **********

%------------------------------------------------------------------------------------------------------------------------------------------------



% ********** PROBLEM 05 **********
\clearpage
\problem [5pts] Prove that there is no positive integer $n$ such that $n^{2} + n^{3} = 100$.

\bigbreak
\textit{Solution:}
    \par Suppose there exists an integer $n$ such that $n^{2} + n^{3} = 100$.
    $\Rightarrow n^{2}(n + 1) = 100$ \\
    $\Rightarrow 100 \vdots n^{2}$ \\
    The set of all factors of 100 is {1, 2, 5, 10, 20, 25, 100}. \\
    $n^{2}$ is a perfect square number \\
    $\Rightarrow n^{2} \in \{1; 25; 100\}.$ \\
    $\Rightarrow n \in \{1; 5; 10\}.$ \\
    $\Rightarrow n^{2}(n + 1) \in \{2; 150; 1100\}.$ \\
    The set does not contain the value 100.
    Conclusion: By contradition, the proposition is proved.

%------------------------------------------------------------------------------------------------------------------------------------------------


% ********** PROBLEM 06 **********
\clearpage
\problem [5pts] Show that if $a$, $b$, and $c$ are real numbers and $a \neq 0$, then there is a
unique solution of the equation $ax + b = c$.

\bigbreak
\textit{Solution:}
\begin{itemize}
    \item Suppose there is no solution for the equation $ax + b = c$. \\
        $\Rightarrow x = \dfrac{c - b}{a}$ is not a real number (conflicting the definition of
        rational numbers)\\
    \item Suppose there is more than one unique solution for the equation $ax + b = c$.
    \par Denote $x_{1}$ and $x_{2}$, ($x_{1} \neq x_{2}$) as two distinct roots of the equation.
        $\Rightarrow ax_{1} + b = ax_{2} + b = 0$ \\
        $\Rightarrow ax_{1} = ax_{2}$ \\
        $\Rightarrow \dfrac{ax_{1}}{a} = \dfrac{ax_{2}}{a}$ \\
        $\Rightarrow x_{1} = x_{2}$ (contradicting the assumption) \\
\end{itemize}
    Conclusion: By contradition, the proposition is proved.

%------------------------------------------------------------------------------------------------------------------------------------------------


% ********** PROBLEM 07 **********
\clearpage
\problem [10pts] Prove the triangle inequality, which states that if $x$ and $y$ are real numbers,
then $|x| + |y| \geq |x + y|$ (where $|x|$ represents the absolute value of $x$, which equals $x$ if
$x \geq 0$ and equals $-x$ if $x < 0$).

\bigbreak
\textit{Solution:}
    \par We have the inequality: $\forall a \in \R, a \leq |a|$. \\
    Similarly:

    \begin{itemize}
        \item \textbf{Case 1}: $|x + y| \geq 0$
        $\Rightarrow |x + y| = x + y \leq x + |y| \leq |x| + |y|$
        \item \textbf{Case 2}: $|x + y| < 0$
        $\Rightarrow |x + y| = - (x + y) = -x - y \leq |-x| - y = |x| - y \leq |x| + |-y| = |x| + |y|$
   \end{itemize}

%------------------------------------------------------------------------------------------------------------------------------------------------


% ********** PROBLEM 08 **********
\clearpage
\problem [15pts] Prove or disprove that if you have an 8-gallon jug of water and two empty jugs with
capacities of 5 gallons and 3 gallons, respectively, then you can measure 4 gallons by successively
pouring some of or all of the water into a jug into another jug.

\bigbreak
\textit{Solution:} \\
\bigbreak

    4 gallons can be measured by doing the following steps:
 \begin{center}
     \begin{tabular}{|c|c|c|c|c|c|}
        \hline
        From & To & Quantity & 8-gallon jug & 5-gallon jug & 3-gallon jug \\ 
        \hline
                     &              &   & 8 & 0 & 0 \\
        \hline
        8-gallon jug & 5-gallon jug & 5 & 3 & 5 & 0 \\
        \hline
        5-gallon jug & 3-gallon jug & 3 & 3 & 2 & 3 \\
        \hline 
        3-gallon jug & 8-gallon jug & 3 & 6 & 2 & 0 \\
        \hline 
        5-gallon jug & 3-gallon jug & 2 & 6 & 0 & 2 \\
        \hline 
        8-gallon jug & 5-gallon jug & 5 & 1 & 5 & 2 \\
        \hline 
        5-gallon jug & 3-gallon jug & 1 & 1 & 4 & 3 \\
        \hline 
    \end{tabular} 
\end{center}
   
%------------------------------------------------------------------------------------------------------------------------------------------------


% ********** PROBLEM 09 **********
\clearpage
\problem [Bonus] Solve as much as you can the exercises in Section 1.7, 1.8 in Rosen's
book, 7th ed. The following are recommended:
    \par Section 1.7: 1, 5, 17, 24, 26, 35
    \par Section 1.8: 13, 17, 20, 22, 30


% ********* EXTRA 1.7 - 1 ********
\clearpage
\textbf{Section 1.7 - Problem 1} Use a direct proof to show that the sum of two odd integers is even.
\bigbreak
\textit{Solution} \\
\bigbreak
    Suppose the two odd integers are $a = 2m + 1$ and $b = 2n + 1$. ($m, n \in \textbf{N}$) \\
    Then: \\
    $a + b = 2m + 2n + 2 = 2(m + n + 1)$
    $\Rightarrow$ The sum of the two odd integers is even.


% ********* EXTRA 1.7 - 5 ********
\clearpage
\textbf{Section 1.7 - Problem 5} Prove that if $m + n$ and $n + p$ are even integers, where $m$, $n$
and $p$ are integers, then $m + p$ is even. What kind of proof did you use?
\bigbreak
\textit{Solution} \\
    Suppose:
    \begin{equation}
        \begin{cases}
            m + n = 2a \\
            n + p = 2b
        \end{cases}
    \end{equation}
    in which $a, b \in \mathbb{N}$. \\
    Then:
    \begin{equation}
        m + n + n + p = m + p + 2n = 2(a + b)
        \Rightarrow m + p + 2n \vdots 2
        \Rightarrow m + p \vdots 2
    \end{equation}
    Conclusion: The proposition is proved using direct proof.

\bigbreak


\end{document}
