\documentclass[10pt]{article}

\usepackage{pictex, latexsym, graphicx,amsmath,amssymb,amsbsy,amsfonts,amsthm,verbatim}
\usepackage{graphics}
\usepackage{fullpage}
\usepackage{fancyhdr}

\usepackage{algorithm,algorithmic}
\usepackage{multirow}

\setlength{\headheight}{12pt}
\setlength{\voffset}{-0.25in}
\setlength{\headsep}{0.25in}
\setlength{\parskip}{1em}
\setlength{\parindent}{0em}

% Counter for the problems.
\newcounter{problem}
\newcommand{\problem}{\textbf{\refstepcounter{problem}Problem \theproblem.} }

% Macros
\def\vu{\mathbf{u}}
\def\vx{\mathbf{x}}
\def\vb{\mathbf{b}}
\def\vv{\mathbf{v}}
\def\vw{\mathbf{w}}

\renewcommand{\implies}{\rightarrow}
\renewcommand{\lor}{\vee}
\renewcommand{\land}{\wedge}
\newcommand{\xor}{\oplus}
\renewcommand{\iff}{\leftrightarrow}
\newcommand{\TRUE}{\mathbf{T}}
\newcommand{\FALSE}{\mathbf{F}}
\newcommand{\universe}{\mathcal{U}}
\newcommand\R{\mathbb{R}}


% Math commands:
% \forall, \exists

\begin{document}
\pagestyle{fancyplain}
%%%%%%%%%%%%%%%%%%%%%%%%%%%%%%%%%%
\chead{DISCRETE STRUCTURE (CO1007) --- SOLUTION to Homework 02 --- PREDICATE LOGIC \& PROOF}
\textit{\textbf{\underline{Instruction:}} Type your answers to the following questions provided by LaTeX and submit a zipped file (included .pdf and .tex files) to E-learning by group (only 4-5 members in each group). Only team leader will submit it. One page per problem. Please use the solution template provided.}

% ********** GROUP INFO **********

\begin{center}
    \begin{tabular}{|c|c|c|c|}
        \hline 
        \multicolumn{4}{|c|}{\textbf{GROUP ... ---- MEMBER LIST}} \\ 
        \hline 
        \textbf{No.} &\qquad\qquad \textbf{Name}\qquad\qquad\qquad & \qquad\textbf{ID}\qquad\qquad & \qquad\textbf{Role}\qquad\qquad \\ 
        \hline 
        1 &  &  &  \\ 
        \hline 
        2 &  &  &  \\ 
        \hline 
        3 &  &  &  \\ 
        \hline 
        4 &  &  &  \\ 
        \hline 
        5 &  &  &  \\ 
        \hline 
    \end{tabular} 
\end{center}

% ************** PROBLEM 15 *******************
\clearpage
\problem [5pts] Find the solution to each of these recurrence relations with the given initial
conditions (Hint: Use an iterative approach).
\begin{enumerate}
    \item $a_{n} = a_{n - 1}, a_{0} = 5$
    \item $a_{n} = a_{n - 1} - n, a_{0} = 4$
    \item $a_{n} = 2a_{n - 1} - 3, a_{0} = - 1$
    \item $a_{n} = (n + 1)a_{n - 1}, a_{0} = 2$
    \item $a_{n} = - a_{n - 1} + n - 1, a_{0} = 7$
\end{enumerate}

\bigbreak
\textit{Solution}
\bigbreak 
\begin{enumerate}
    \item $a_{n} = a_{n - 1}, a_{0} = 5$
        \begin{itemize}
            \item $a_{0} = 5$
            \item $a_{1} = a_{0} = 5$
            \item $a_{2} = a_{1} = 5$
            \item \ldots
        \end{itemize}
        $\Rightarrow a_{n} = 5$

    \item $a_{n} = a_{n - 1} - n, a_{0} = 4$
        \begin{align*}
            a_{0} &=& 4  \\
            a_{1} &=& a_{0} - 1 &=& 4 - 1 &=& 3 &=& 4 - 1 \\
            a_{2} &=& a_{1} - 2 &=& 3 - 2 &=& 1 &=& 4 - 1 - 2 \\
            a_{3} &=& a_{2} - 3 &=& 1 - 3 &=& -2 &=& 4 - 1 - 2 - 3 \\
            a_{4} &=& a_{3} - 4 &=& -2 - 4 &=& -6 &=& 4 - 1 - 2 - 3 - 4 \\
        \end{align*}
        $\Rightarrow a_{n} = 4 - \dfrac{n(n + 1)}{2}$

    \item $a_{n} = 2a_{n - 1} - 3, a_{0} = - 1$
        \begin{align*}
            a_{0} &=& -1 \\
            a_{1} &=& 2a_{0} - 3 &=& 2(-1) - 3 &=& -5 &=& -1 - 4 &=& -1 - 4 \\
            a_{2} &=& 2a_{1} - 3 &=& 2(-5) - 3 &=& -13 &=& -1 - 4 - 8 &=& -1 - 4 - 4.2 \\
            a_{3} &=& 2a_{2} - 3 &=& 2(-13) - 3 &=& -29 &=& -1 - 4 - 8 - 16 &=&
                -1 - 4 - 4.2 - 4.2^{2} \\
            a_{4} &=& 2a_{3} - 3 &=& 2(-29) - 3 &=& -61 &=& -1 - 4 - 8 - 16 - 32 &=&
                -1 - 4 - 4.2 - 4.2^{2} - 4.2^{3} \\
        \end{align*}
        $\Rightarrow a_{n} = -1 - \sum_{i = 1}^{n} 4^{i} = -1 - \dfrac{4(\,2^{n + 1} - 1)\,}{2 - 1}$
            $ = -1 - 4(\,2^{n + 1} - 1)\,$

    \item $a_{n} = (n + 1)a_{n - 1}, a_{0} = 2$
        \begin{align*}
            a_{0} && && && &=& 2.1 \\
            a_{1} &=& (1 + 1).a_{0} &=& 2.2 &=& 4 &=& 2.1.2 \\
            a_{2} &=& (2 + 1).a_{1} &=& 3.4 &=& 12 &=& 2.1.2.3 \\
            a_{3} &=& (3 + 1).a_{2} &=& 4.12 &=& 48 &=& 2.1.2.3.4 \\
        \end{align*}
        $\Rightarrow a_{n} = 2.(n + 1)!$

    \item $a_{n} = - a_{n - 1} + n - 1, a_{0} = 7$
        \begin{align*}
            a_{0} &=& 7 && &=& 7.(-1)^{0} + \lfloor \dfrac{0}{2} \rfloor \\
            a_{1} &=& -7 + 1 - 1 &=& -7 &=& 7.(-1)^{1} + \lfloor \dfrac{1}{2} \rfloor \\
            a_{2} &=& 7 + 2 - 1  &=& 8  &=& 7.(-1)^{2} + \lfloor \dfrac{2}{2} \rfloor \\
            a_{3} &=& -8 + 3 - 1 &=& -6 &=& 7.(-1)^{3} + \lfloor \dfrac{3}{2} \rfloor \\
            a_{4} &=& 6 + 4 - 1  &=& 9  &=& 7.(-1)^{4} + \lfloor \dfrac{4}{2} \rfloor \\
        \end{align*}
        $\Rightarrow a_{n} = 7.(-1)^{n} + \lfloor \dfrac{n}{2} \rfloor$

\end{enumerate}

% ********* EXTRA 2.1 - 9 ********
\clearpage
\textbf{Section 2.1 - Problem 9} Determine wheter each of these statements is true or false.
\bigbreak
\textit{Solution} 
\bigbreak

    \par a. $0 \in \emptyset$: False
    \par b. $\emptyset \in \{0\}$: False
    \par c. $\{0\} \subset \emptyset$: False
    \par d. $\emptyset \subset \{0\}$: True
    \par e. $\{0\} \in \{0\}$: False
    \par f. $\{0\} \subset \{0\}$: True
    \par g. $\{\emptyset\} \subseteq \{\emptyset\}$: True

% ********* EXTRA 2.1 - 10 ********
\clearpage
\textbf{Section 2.1 - Problem 10} Determine wheter these statements are true or false.
\bigbreak
\textit{Solution} 
\bigbreak

    \par a. $\emptyset \in \{\emptyset\}$: True
    \par b. $\emptyset \in \{\emptyset, \{\emptyset\}\}$: True
    \par c. $\{\emptyset\} \in \{\emptyset\}$: False
    \par d. $\{\emptyset\} \in \{\{\emptyset\}\}$: True
    \par e. $\{\emptyset\} \subset \{\emptyset, \{\emptyset\}\}$: True
    \par f. $\{\{\emptyset\}\} \subset \{\emptyset, \{\emptyset\}\}$: True
    \par g. $\{\{\emptyset\}\} \subset \{\{\emptyset\}, \{\emptyset\}\}$: True

% ********* EXTRA 2.1 - 22 ********
\clearpage
\textbf{Section 2.1 - Problem 22} Can you conclude that $A = B$ if $A$ and $B$ are two sets
with the same power set?
\bigbreak
\textit{Solution} 
\bigbreak

    Suppose $A$ and $B$ are two sets with different power sets. \\
    $\Rightarrow (\exists x \in A, x \not \in B) \lor (\exists x \in B, x \not \in A)$ \\
    $\Rightarrow A \neq B$ \\
    Conclusion: The proposition is proved by contraposition.

% ********* EXTRA 2.1 - 26 ********
\clearpage
\textbf{Section 2.1 - Problem 26} Show that if $A \subseteq C$ and $B \subseteq D$, then
$A \times B \subseteq C \times D$.
\bigbreak
\textit{Solution} 
\bigbreak

    Let $P \equiv A \subseteq C \land B \subseteq D$ \\
    Let $Q \equiv A \times B \subseteq C \times D$ \\
    Suppose $\lnot Q$ \\
    $\Rightarrow A \times B \not \subseteq C \times D $ \\
    $\Rightarrow \exists (x, y) \in A \times B: (x, y) \not \in C \times D$ \\
    $\Rightarrow \exists x, y: (x \in A \land x \not \in C) \lor (y \in B \land y \not \in D)$ \\
    $\Rightarrow (A \not \subset C) \lor (B \not \subset D)$ \\
    $\Rightarrow \lnot P$ \\

    Conclusion: The proposition is proved by contraposition.

% ********* EXTRA 2.1 - 29 ********
\clearpage
\textbf{Section 2.1 - Problem 29} What is the Cartesian product $A \times B \times C$ where $A$
is the set of all airlines and $B$ and $C$ are both the set of all cities in the United States?
Give an example of how this Cartesian product can be used.
\bigbreak
\textit{Solution} 
\bigbreak

    The product is the set in which every element is a flight supported by a specific airline
    from a specific city to another city. \\
    Example of how this Carteisan product can be used: It can help a customer to choose a good
    airline for his trip from his current city to a certain destination.

% ********* EXTRA 2.1 - 40 ********
\clearpage
\textbf{Section 2.1 - Problem 40} Explain why $(A \times B) \times (C \times D)$ and
$A \times (B \times C) \times D$ are not the same.
\bigbreak
\textit{Solution} 
\bigbreak

    $(A \times B) \times (C \times D)$ \\
    $= \{(a, b) | a \in A, b \in B\} \times \{(c, d) | c \in C, d \in D\} $ \\
    $= \{((a, b), (c, d)) | a \in A, b \in B, c \in C, d \in D\}$ \\
    \break
    $A \times (B \times C) \times D$ \\
    $= A \times \{(b, c) | b \in B, c \in C\} \times D$ \\
    $= \{(a, (b, c), d) | a \in A, b \in B, c \in C, d \in D\}$ \\
    \break
    Conclusion: $(A \times B) \times (C \times D)$ and $A \times (B \times C) \times D$
    are not the same.


% ********* EXTRA 2.2 - 2 ********
\clearpage
\textbf{Section 2.2 - Problem 2} Suppose that $A$ is the set of sophomores at your school
and $B$ is the set of students in discrete mathematics at your school. Express each of these
sets in terms of $A$ and $B$.

\bigbreak
\textit{Solution} 
\bigbreak

    \par (a) The set of sophomores taking discrete mathematics in your school: $A \cap B$
    \par (b) The set of sophomores at your school who are not taking discrete mathematics:
    $A \cap \overline{B}$
    \par (c) The set of students at your school who either are sophomores or are taking
    discrete mathematics: $A \cup B$
    \par (d) The set of students at your school who either are not sophomores or are not
    taking discrete mathematics: $\overline{A} \cup \overline{B} \equiv \overline{A \cap B}$

% ******** EXTRA 2.2 - 30 ******** 
\clearpage
\textbf{Section 2.2 - Problem 30} Can you conclude that $A = B$ if $A$, $B$ and $C$ are sets
such that:
\par (a) $A \cup C = B \cup C$
\par (b) $A \cap C = B \cap C$
\par (c) $A \cup C = B \cup C \mbox{ and } A \cap C = B \cap C$
\bigbreak
\textit{Solution} 
\bigbreak

    \par \textbf{(a)} $A \cup C = B \cup C \implies A = B$ ? \\
        Suppose $A = \emptyset$, $B = {1}$, $C = {1}$. \\
        We have: $A \cup C = B \cup C = {1}$. However, $A \neq B$. \\
        Conclusion: The proposition is proved wrong using counterexample.

    \par \textbf{(b)} $A \cap C = B \cap C \implies A = B$ ? \\
        Suppose $A = {1, 2}$, $B = {1, 2, 3}$, $C = {1}$. \\
        We have: $A \cup C = B \cup C = {1}$. However, $A \neq B$. \\
        Conclusion: The proposition is proved wrong using counterexample.

    \par \textbf{(c)} $(A \cup C = B \cup C) \land (A \cap C = B \cap C) \implies A = B$ ? \\
        Let
        \begin{equation}
            \begin{cases}
            P \equiv (A \cup C = B \cup C) \land (A \cap C = B \cap C) \\ 
            Q \equiv (A \cup C = B \cup C) \land (A \cap C = B \cap C) \\ 
            \end{cases}
        \end{equation}
        Suppose $\lnot Q \equiv A \neq B$ \\
        $\Rightarrow \exists x, (x \in A \land x \not \in B) \lor (x \not \in A \land x \in B)$ \\
        $\Rightarrow \exists x, (x \in A \xor x \in B)$ \\
        We also have: $x \in C \lor x \not \in C$ \\
        \textbf{Case 1}: $x \not \in C$ \\
        $\Rightarrow \exists x, (x \in A \xor x \in B) \land (x \not \in C)$ \\
        $\Rightarrow A \cap C \neq B \cap C$ \\
        $\Rightarrow \lnot P$ \\
        Conclusion: The proposition is proved correct by contraposition.

% ******** EXTRA 2.2 - 41 ******** 
\clearpage
\textbf{Section 2.2 - Problem 41}
Suppose that A, B, and C are sets such that $A \xor C$ = $B \xor C$. Must it
be the case that $A = B$?
\bigbreak
\textit{Solution} 
\bigbreak
    Based on the problem \textbf{2.2 - 30 - (c)}: \\
    $(A \cup C = B \cup C) \land (A \cap C = B \cap C) \implies A = B$ \\
    $\Rightarrow (A \xor C = B \xor C) \implies A = B$

% ******** EXTRA 2.2 - 46 ******** 
\clearpage
\textbf{Section 2.2 - Problem 46}
Show that if $A$, $B$ and $C$ are sets, then
\begin{equation}
    |A \cup B \cup C| = |A| + |B| + |C| - |A \cap B| - |A \cap C| - |B \cap C|
    + |A \cap B \cap C|
\end{equation}
\bigbreak
\textit{Solution} 
\bigbreak
    First, we try to prove the lemma: $|A \cup B| = |A| + |B| - |A \cap B|$ \\
    The cardinality of a set is the number of distinct element in that set.
    The sum $|A| + |B|$ has already taken in the number of overlapping elements in
    $A \cap B$ twice. Therefore, we have to subtract $|A \cap B|$.
    \break

    Back to the given proposition. Let $D = B \cup C$.
    \begin{align*}
        LHS & = |A \cup B \cup C| \\
        & = |A \cup D| \\
        & = |A| + |D| - |A \cap D| \\
        & = |A| + |B \cup C| - |A \cap (B \cup C)| \\
        & = |A| + |B \cup C| - |(A \cap B) \cup (A \cap C)| \\
        & = |A| + |B| + |C| - |B \cup C| - |A \cap B| - |A \cap C| + |A \cap B \cap C| \\
        & = RHS
    \end{align*}
    Conclusion: the proposition is proved.


% ******** EXTRA 2.2 - 61 ******** 
\clearpage
\textbf{Section 2.2 - Problem 61}
Let $A$ and $B$ be the multisets $\{3 \cdot a, 2 \cdot b, 1 \cdot c\}$ and
$\{2 \cdot a, 3 \cdot b, 4 \cdot d\}$, respectively. Find:
\par \textit{a.} $A \cup B$
\par \textit{b.} $A \cap B$
\par \textit{c.} $A - B$
\par \textit{d.} $B - A$
\par \textit{e.} $A + B$
\bigbreak
\textit{Solution} 
\bigbreak

\par \textit{a.} $A \cup B = \{3 \cdot a, 3 \cdot b, 1 \cdot c, 4 \cdot d\}$
\par \textit{b.} $A \cap B = \{2 \cdot a, 2 \cdot b\}$
\par \textit{c.} $A - B = \{1 \cdot a, 1 \cdot c\}$
\par \textit{d.} $B - A = \{1 \cdot b, 4 \cdot c\}$
\par \textit{e.} $A + B = \{5 \cdot a, 5 \cdot b, 1 \cdot b, 4 \cdot c\}$

% ******** EXTRA 2.2 - 62 ******** 
\clearpage
\textbf{Section 2.2 - Problem 62}
Suppose that A is the multiset that has as its elements the types of computer
equipment needed by one department of a university and the multiplicities are
the number of pieces of each type needed, and B is the analogous multiset for
a second department of the university. For instance, A could be the multiset
{107 $\cdot$ personal computers, 44 $\cdot$ routers, 6 $\cdot$ servers} and B could be
the multiset {14 $\cdot$ personal computers, 6 $\cdot$ routers, 2 $\cdot$ mainframes}.

\par \textit{a.} What combination of A and B represents the equipment the university
should buy assuming both departments use the same equipment?
\par \textit{b.} What combination of A and B represents the equipment that will be
used by both departments if both departments use the same equipment?
\par \textit{c.} What combination of A and B represents the equipment that the second
department uses, but the first department does not, if both departments use the same
equipment?
\par \textit{d.} What combination of A and B represents the equipment that the
university should purchase if the departments do not share equipment?

\bigbreak
\textit{Solution} 
\bigbreak

\par \textit{a.} $A \cup B$
\par \textit{b.} $A \cap B$
\par \textit{c.} $B - A$
\par \textit{d.} $A \cap B$


\end{document}
