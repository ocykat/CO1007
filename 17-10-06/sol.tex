\documentclass[10pt]{article}

\usepackage{pictex, latexsym, graphicx,amsmath,amssymb,amsbsy,amsfonts,amsthm,verbatim}
\usepackage{graphics}
\usepackage{fullpage}
\usepackage{fancyhdr}

\usepackage{algorithm,algorithmic}
\usepackage{multirow}

\setlength{\headheight}{12pt}
\setlength{\voffset}{-0.25in}
\setlength{\headsep}{0.25in}
\setlength{\parskip}{1em}
\setlength{\parindent}{0em}

% Counter for the problems.
\newcounter{problem}
\newcommand{\problem}{\textbf{\refstepcounter{problem}Problem \theproblem.} }

% Macros
\def\vu{\mathbf{u}}
\def\vx{\mathbf{x}}
\def\vb{\mathbf{b}}
\def\vv{\mathbf{v}}
\def\vw{\mathbf{w}}

\renewcommand{\implies}{\rightarrow}
\renewcommand{\lor}{\vee}
\renewcommand{\land}{\wedge}
\newcommand{\xor}{\oplus}
\renewcommand{\iff}{\leftrightarrow}
\newcommand{\TRUE}{\mathbf{T}}
\newcommand{\FALSE}{\mathbf{F}}
\newcommand{\universe}{\mathcal{U}}
\newcommand\R{\mathbb{R}}


% Math commands:
% \forall, \exists

\begin{document}
\pagestyle{fancyplain}
%%%%%%%%%%%%%%%%%%%%%%%%%%%%%%%%%%
\chead{DISCRETE STRUCTURE (CO1007) --- SOLUTION to Homework 02 --- PREDICATE LOGIC \& PROOF}
\textit{\textbf{\underline{Instruction:}} Type your answers to the following questions provided by LaTeX and submit a zipped file (included .pdf and .tex files) to E-learning by group (only 4-5 members in each group). Only team leader will submit it. One page per problem. Please use the solution template provided.}

% ********** GROUP INFO **********

\begin{center}
    \begin{tabular}{|c|c|c|c|}
        \hline 
        \multicolumn{4}{|c|}{\textbf{GROUP ... ---- MEMBER LIST}} \\ 
        \hline 
        \textbf{No.} &\qquad\qquad \textbf{Name}\qquad\qquad\qquad & \qquad\textbf{ID}\qquad\qquad & \qquad\textbf{Role}\qquad\qquad \\ 
        \hline 
        1 &  &  &  \\ 
        \hline 
        2 &  &  &  \\ 
        \hline 
        3 &  &  &  \\ 
        \hline 
        4 &  &  &  \\ 
        \hline 
        5 &  &  &  \\ 
        \hline 
    \end{tabular} 
\end{center}

% ************** PROBLEM 06 *******************
\clearpage
\problem [5pts] A domino is a flat rectangular block the face of which is divided into two square
parts, each part showing from zero to six pips (or dots). Playing a game consists of playing
dominoes with a matching number of pips. Explain why there are 28 dominoes in a complete set.

\bigbreak
\textit{Solution}
\bigbreak

\par Denote $X$ as the number of dominoes in a complete set.
\par Let $f(k) = |S_{k}|$ where $S_{k} = \{(a, k)| a \leq k\}$. \\
$\Rightarrow f(k) = k + 1$ \\
$\Rightarrow X = \sum_{k = 0}^{6} f(k) = \sum_{k = 0}^{6} k + 1 = 1 + 2 + \ldots + 7$ \\
$\Rightarrow X = \dfrac{7 \cdot 8}{2} = 28$


% ************** PROBLEM 07 *******************
\clearpage
\problem [10pts]
\bigbreak
\textit{Solution}
\bigbreak

\begin{enumerate}
    \item How many solutions are there to the equation $x_{1} + x_{2} + x_{3} + x_{4} \leq 20$, where
    $x_{1}, x_{2}, x_{3}, x_{4}$ are non-negative integers? are positive integers?

    \par Rephrase the problem into ``How many ways are there to arrange 3 vertical bars and 
    $\leq 20$ balls together.

    \par Rephrase the problem \textbf{again} into ``How many ways are there to arrange 4 vertical bars
    and \textbf{exactly} 20 balls together."

    \begin{itemize}
        \item Case 1: $x_{1}, x_{2}, x_{3}, x_{4}$ are non-negative integers.\\
        $\Rightarrow$ The vertical bars can be put next to each other.
        \par There are (20 + 4) = 24 objects. Thus, there are 24 positions.
        $\Rightarrow$ There are $C_{4}^{20 + 4} = 10626$ ways to arrange.

        \item Case 2: $x_{1}, x_{2}, x_{3}, x_{4}$ are positive integers.\\
        \par To solve this problem, we set up 4 (ball, bar) pairs (ball first, bar second) to
        guarantee that the bars cannot be next to each other.
        \par There are (20 + 4 - 4) = 20 objects. Thus, there are 20 positions.
        $\Rightarrow$ There are $C_{4}^{20} = 4845$ ways to arrange.
    \end{itemize}

    \item How many solutions are there to the equation $x_{1} + x_{2} + x_{3} + x_{4} = 20$, where
    $x_{1}, x_{2}, x_{3}, x_{4}$ are non-negative integers with $x_{1} \leq 3$, $x_{2} \geq 2$,
    $x_{3} > 4$?

    \par Rephrase the problem into ``How many ways are there to arrange 3 vertical bars and 20 balls
        together" satisfying the mentioned conditions.
    \par Since $x_{2} \geq 2$, we group (2 balls, bar 2) together into group 2 (the balls come first).
    \par Since $x_{3} > 4$, we group (5 balls, bar 3) together into group 3 (the balls come first).
    \par If $x_{1} = i$ ($i \leq 3$), we group ($i$ ball(s), bar 1) together into group 1. (the
        balls come first). Also, \textbf{the position of group 1 has to be fixed on the leftmost
        side} so that the less than sign can be satisfied.
    \par The number of object when $x_{1} = i$ is: 20 + 3 - i - 2 - 5 = 16 - i. In addition, the
        position of group 1 has been fixed. As a result, the number of arrangements is
        $C_{4 - 1}^{16 - i - 1} = C_{3}^{15 - i} $. 
    \par In total, the number of arrangements is:
    \begin{align*}
        \sum_{i = 0}^{3} C_{3}^{15 - i} = C_{3}^{15} + C_{3}^{14} + C_{3}^{13} + C_{3}^{12} = 1325
    \end{align*}

    \item How many ways to arrange 30 marbles in 5 different boxes, so that box 1 has at least 5
        balls, knowing that box 2 and box 3 do not contain more than 6 balls.

    \par Applying the same way of reasoning like question number 2 based on grouping.
        Note that group 2 and 3 are fixed on the leftmost side and always stay next to each other.
        We can come up with the formula:
    \par Number of arrangement
    \begin{align*}
        & \sum_{i = 0}^{6} \sum_{j = 0}^{6} C_{3}^{28 - i - j} \\
        = & \sum_{i = 0}^{6} \bigg[ C_{3}^{28 - i} + C_{3}^{27 - i} + \ldots
        + C_{3}^{22 - i} \bigg] \\
        = & 16436 + 14490 + 12705 + 11074 + 9590 + 8246 + 7035 \\
        = & 79576
    \end{align*}

\end{enumerate}

% ************** PROBLEM 08 *******************
\clearpage
\problem [10pts] Given a set $A = \{1, 2, 3, 4, 5, 6, 7, 8, 9\}$.
\begin{enumerate}
    \item From $A$ we can create how many different numbers containing 5 digits such that the middle
        number of this sequence is divisible by 5, number 5 appears only once, and the last number
        is an odd number?
    \item From $A$ we can create how many different numbers containing 6 digits such that the
        odd numbers cannot stand side by side (an odd number is not next to an odd number)?
    \item We can create how many odd numbers (a number includes 6 digits from $A$) such that the
        number 5 always appear twice?
\end{enumerate}
\bigbreak
\textit{Solution}
\bigbreak

\begin{enumerate}
    \item There is/are:
        \begin{itemize}
            \item 9 ways to choose digit 1.
            \item 9 ways to choose digit 2.
            \item 1 ways to choose digit 3.
            \item 9 ways to choose digit 4.
            \item 5 ways to choose digit 5. (1, 3, 5, 7, 9)
        \end{itemize}
        $\Rightarrow$ There are: $9^{3} \cdot 5 = 3645$ different numbers.

    \item Divide the question into 2 cases: \\
        \textbf{Case 1}: Digit 1 is an odd number. There are
        \begin{itemize}
            \item 5 ways to choose digit 1.
            \item 4 ways to choose digit 2. (even)
            \item 5 ways to choose digit 3. (odd)
            \item 4 ways to choose digit 4. (even)
            \item 5 ways to choose digit 5. (odd)
            \item 4 ways to choose digit 6. (even)
        \end{itemize}
        \textbf{Case 2}: Digit 1 is an even number. There are
        \begin{itemize}
            \item 4 ways to choose digit 1.
            \item 5 ways to choose digit 2. (even)
            \item 4 ways to choose digit 3. (odd)
            \item 5 ways to choose digit 4. (even)
            \item 4 ways to choose digit 5. (odd)
            \item 5 ways to choose digit 6. (even)
        \end{itemize}
        \textbf{Conclusion}: There are: $2 \cdot 4^{3} \cdot 5^{3} = 16000$ different numbers
        in total.

    \item Divide the question into 2 cases: \\
        \textbf{Case 1}: The last digit is 5.
        \begin{itemize}
            \item There are 5 possible positions for the remaining number 5.
            \item For the other 4 digits, each will have 9 ways to choose.
        \end{itemize}
        $\Rightarrow$ There are: $5 \cdot 9^{4} = 32805$ different numbers. \\

        \textbf{Case 2}: The last digit is not 5.
        \begin{itemize}
            \item There are $C_{2}^{5}$ possible positions for two the two number-5s.
            \item There are 5 ways to choose digit 5.
            \item For the other 3 digits, each will have 9 ways to choose.
        \end{itemize}
        $\Rightarrow$ There are: $C_{2}^{5} \cdot 5 \cdot 9^{3} = 36450$ different numbers. \\
        \textbf{Conclusion:} There are $32805 + 36450 = 69255$ different numbers.
\end{enumerate}


\end{document}
